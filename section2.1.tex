%!TEX root = ./main.tex
%!TEX encoding = UTF-8 Unicode

\section{Single-Particle Quantum Mechanics and Path Integrals}

Everybody who learnt quantum mechanics should have heard about the interference experiment of electron waves through slits. Figure \ref{Fig2.1} shows the principle. Electrons emitted from an electron gun pass either through slit A or slit B of a shield and finally illuminate a screen. Because it is possible to avoid more than one electron reaching the screen at the same time by adjusting the intensity of the electron gun to be small enough, it is clear that electrons can indeed be interpreted as propagating “particles”. However, when this kind of experiment is performed over a long period, the observed distribution of electrons at all points of the screen becomes the interference pattern of a wave.

In quantum mechanics, this interference pattern is explained as follows. We call the paths from the electron gun through slit A or slit B to the point P on the screen $P _ { \mathrm { A } }$ or $P _ { \mathrm { B } } ,$ respectively. $P _ { \mathrm { A } }$ and $P _ { \mathrm { B } }$ each corresponds to a complex amplitude $a _ { \mathrm { A } }$ and $a _ { \mathrm { B } }$ of a quantum mechanical wave. Then, the phase difference $\varphi$ of the complex functions $a _ { \mathrm { A } }$ and $a _ { \mathrm { B } } \left( a _ { \mathrm { A } } / a _ { \mathrm { B } } \propto \mathrm { e } ^ { \mathrm { i } \varphi } \right)$ equals the phase difference of the waves. The complex amplitude corresponding to the process that an electron started from the electron gun and reached the point P, without asking whether the electron passed through slit A or slit B is given by the sum of the amplitudes $P _ { \mathrm { A } }$ and $P _ { \mathrm { B } }$. The intensity of the wave reaching P (in quantum mechanics the probability of reaching P) is given by the absolute value of the square of the complex amplitude
\[\left| a _ { \mathrm { A } } + a _ { \mathrm { B } } \right| ^ { 2 } = \left| a _ { \mathrm { A } } \right| ^ { 2 } + \left| a _ { \mathrm { B } } \right| ^ { 2 } + 2 \left| a _ { \mathrm { A } } \right| \left| a _ { \mathrm { B } } \right| \cos \varphi\]
This expression varies periodically depending on $\varphi$. 


































































































