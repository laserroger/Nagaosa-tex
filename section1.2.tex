%!TEX root = ./main.tex
%!TEX encoding = UTF-8 Unicode

\section[Many-Particle Quantum Mechanics: Second Quantization]{Many-Particle Quantum Mechanics: \\ Second Quantization}

In this section, we consider the many-particle case. In this case the wave function is a function of the time $t$ and $3N$-dimensional coordinate space (for a moment, we omit the spin dependence)
\be
\langle\bm r_1,\cdots,\bm r_N|\psi(t)\rangle=\psi(\bm r_1,\cdots,\bm r_N,t)
\ee
Unlike classical mechanics, in many-particle quantum mechanics it is impossible in principle to distinguish particles of the same species. We cannot think about indistinguishable particles as rigid bodies; however, it should be possible to get an idea of it with the following metaphor. \\
\indent Think about a luminous advertisement screen. By switching the lamps on and off at every point of the surface, it is possible to create a moving picture. Places that are illuminated have more energy than the other places, and therefore there should be a particle. A state with $N$ particles at $\bm r_1\cdots\bm r_N$ should correspond to the state where $N$ lights are illuminated. In this metaphor, it is clear that it is not possible to distinguish the particles. The particle appears as an illuminated lamp, and it is not possible to trace back the way of it as rigid body. 

In mathematical language, this means that exchanging the order of $\bm r_1\cdots\bm r_N$ does not lead to a new state, but should lead to the very same state again. Explicitly, taking care also of the statistics when exchanging $\bm r_i$ with $\bm r_j$, we obtain
\be\begin{split}
\psi(\bm r_1,&\cdots,\bm r_j,\cdots,\bm r_i,\cdots,\bm r_N)\\
=&\begin{cases}
+\psi(\bm r_1,&\cdots,\bm r_i,\cdots,\bm r_j,\cdots,\bm r_N)\quad  (\text{boson})\\
-\psi(\bm r_1,&\cdots,\bm r_i,\cdots,\bm r_j,\cdots,\bm r_N)\quad  (\text{fermion})
\end{cases}. 
\end{split}\ee
However, as the reader might have realized, we now have a little problem with the interpretation of the wave function. Of course, 
\be
P(\bm r_1,\cdots,\bm r_N;t)=|\psi(\bm r_1,\cdots,\bm r_N;t)|^2
\ee
is the probability of finding at time $t$ the $N$ particles at $\bm r_1\cdot\bm r_N$. However, the image that we have in mind in the single-particle case, namely that $\psi(\bm r,t)$ is the complex wave amplitude at the position $\bm r$ in the three0dimensional physical space, is ruined because we now have to think mathematically about a $3N$-dimensional space. The answer to the question whether in the many-particle case it is still possible to think about a wave function in the physical three-dimensional space is given by the so-called method of second quantization. 

For a detailed discussion of the second quantization the reader is referred to \cite{3}. Here, we proceed in a heuristic way. Let us return to the single-particle case. We decompose the single-particle wave function $\psi(\bm r,t)$ is an orthonormal basis $\phi_n(\bm r)$
\be\label{eq1.2.4}
\psi(\bm r,t)=\sum_na_n(t)\phi_n(\bm r)
\ee
The whole time dependence is given by the expansion coefficients $a_n(t)$. Inserting \eqref{eq1.2.4} into the Schrödinger equation, we obtain
\be
\ii\hbar\sum_n\frac{\dd a_n(t)}{\dd t}\phi_n(\bm r)=\sum_na_n(t)\hat{H}\phi_n(\bm r)
\ee
Multiplying by $\phi_n^*(r)$ and integrating over $\bm r$, we obtain
\be\label{eq1.2.6}
\ii\hbar\frac{\dd a_n(t)}{\dd t}=\sum_m\langle\phi)n|\hat{H}|\phi_m\rangle a_m(t).
\ee
The complex conjugate of this equation is given by
\be\label{eq1.2.7}
\ii\hbar\frac{\dd a_n^*(t)}{\dd t}=-\sum_m a_m^*(t)\langle\phi_m|\hat{H}|\phi_n\rangle, 
\ee
where $\hat{H}^\dagger=\hat{H}$ has been used. 

These equations determine the time development of the expansion coefficients $a_n(t)$. We will now modify them a little. In order to do so, we must think about the energy expectation value $\hat H$ in the state $|\psi(t)\rangle$:
\be\label{eq1.2.8}
\langle\hat{H}\rangle=\langle\psi(t)|\hat{H}|\psi(t)\rangle=\sum_{n,m}a_n^*(t)a_m(t)\langle\phi_n|\hat{H}|\phi_m\rangle
\ee
Using this expression we can write for \eqref{eq1.2.6} and \eqref{eq1.2.7}, respectively
\be\label{eq1.2.9}
\frac{\dd a_n(t)}{\dt}=\frac{\partial\langle\hat{H}\rangle}{\partial(\ii\hbar a_n^*)}
\ee
and
\be\label{eq1.2.10}
\frac{\dd(\ii\hbar a_n^*(t))}{\dt}=-\frac{\partial\langle\hat H\rangle}{\partial a_n}.
\ee
Here, we see that these equations are formally analogous to Hamilton's canonical equation with the correspondences $a_n\leftrightarrow x$ and $\ii\hbar a_n^*\leftrightarrow p$. However, of course the espansion coefficients $a_n$ and $a_n^*$ are not dynamical variables of the system. 

The idea of second quantization is to promote $a_n$ and $a_n^*$ to operators and to interpret $N_n=N|a_n|^2=Na_na_n^*$ as physical quantities of the system. Originally, in single-particle quantum mechanics, $N_n$ is $N$ times the probability of detecting the particle when the same experiment is performed $N$ times. Then, $N$ is the number of experiments, and therefore in principle this experiment can be performed in single particle quantum mechanics. 

On the other hand, consider a system with $N$ non-interacting particles where only one experiment is performed, and where the number of particles in the single particle state $n$ is observed to be $\hat{N}_n$. Since both experiments described above are different, $N_n$ and $\hat{N}_n$ are different quantities - repeating number, or observed number of particles. Experience tells us that these two numbers often agree. Admitting this, the number $N_n$ appearing in the single particle system after performing $N$ experiments will be promoted to the observable (physical quantity) $\hat N=\hat{A}_n^\dagger\hat{A}_n$ in the $N$-particle system. At the same time, 
\be\begin{split}
\sqrt{N}a_n&\to\hat{A}_n,\\
\sqrt{N}a_n^*&\to\hat{A}_n^\dagger
\end{split}\ee
are promoted to operators. Taking into account also that the energy expectation value $\hat{H}$ is multiplied by $N$, it is clear from \eqref{eq1.2.9} and \eqref{eq1.2.10} that $\hat{A}_n$ and $\ii\hbar\hat{A}_n^*$ are canonical conjugate variables. Therefore, we suppose that they fulfill the same commutation relation as $\hat{\bm r}$ and $\hat{\bm p}$:
\be\label{eq1.2.12}
[\hat{A}_n,\ii\hbar\hat{A}_n^\dagger]=\ii\hbar
\ee
Further generalizing \eqref{eq1.2.12}, we obtain
\be\label{eq1.2.13}\begin{split}
[\hat{A}_n,\ii\hbar\hat{A}_m^\dagger]=\ii\hbar\delta_{n,m},\\
[[\hat{A}_n,\hat{A}_m]=[\ii\hbar\hat{A}_n^\dagger,\ii\hbar\hat{A}_m^\dagger]=0.
\end{split}\ee
From these commutation relations it can be understood that $\hat{A}_j^\dagger$, and $A_n$ are creating and annihilating one particle in the state $n$, respectively. We write $|N_n\rangle$ for the many-particle state where $N_n$ particles are in the single-particle state $n$. Then, the following equation holds:
\be
\hat{N}_n|N_n\rangle=N_n|N_n\rangle
\ee
Acting with $\hat{N}_n$ on the state $A_n^\dagger|N_n\rangle$, we obtain
\be\label{eq1.2.15}\begin{split}
\hat{N}_n\hat{A}_n^\dagger|N_n\rangle&=([\hat{N}_n,\hat{A}_n^\dagger]+\hat{A}_n^\dagger\hat{N}_n)|N_n\rangle\\
&=(N_n+1)\hat{A}_n^\dagger|N_n\rangle
\end{split}\ee
Here, we used a variation of equation \eqref{eq1.2.13}
\be\tag{1.2.13'}
[\hat{N}_n,\hat{A}_n^\dagger]=[\hat{A}_n^\dagger\hat{A}_n,\hat{A}_n^\dagger]=\hat{A}_n^\dagger[\hat{A}_n,\hat{A}_n^\dagger]=\hat{A}_n^\dagger
\ee
From \eqref{eq1.2.15} it follows that the particle number eigenvalue of the state AjJlVn) is given by $N_n + 1$, and therefore $\hat{A}_n^\dagger$, is an operator that increases the particle number by one. In the same way, it can be seen that An is an operator that decreases the particle number by one. Starting from the state $|N_n = 0\rangle$, we can create the states $N_n = 0,1,2,\cdots$ by acting on it successively with $\hat{A}^\dagger_n$. The particle picture arises because the particle number eigenvalue $N_n$ counts discrete integer numbers.

On the other hand, what might the wave picture be? In order to understand it, we define the field operators $\hat{\psi}(\bm r)$ and $\hat{\psi}^\dagger(\bm r)$
\be\label{eq1.2.16}\begin{split}
\hat{\psi}(\bm{r})=\sum_n\hat{A}_n\phi_n(\bm{r}),\\
\hat{\psi}^\dagger(\bm{r})=\sum_n\hat{A}^\dagger_n\phi_n(\bm{r}),
\end{split}\ee
Using the commutator relation \eqref{eq1.2.13} and the fact that $\{\phi_n(\bm r)\}$ is an orthonormal basis of single-particle states, we obtain
\be\label{eq1.2.17}\begin{split}
[\hat{\psi}(\bm{r}),\hat{\psi}^\dagger(\bm{r}')]&=\sum_{n,m}[\hat{A}_n,\hat{A}_m^\dagger]\phi_n(\bm{r})\phi_m^*(\bm{r}')\\
&=\sum_n\langle\bm{r}|n\rangle\langle n|\bm{r}'\rangle=\langle \bm{r}|\bm{r}'\rangle=\delta(\bm{r}-\bm{r}')
\end{split}\ee
and
\be\label{eq1.2.18}
[\hat{\psi}(\bm{r}),\hat{\psi}(\bm{r}')]=[\hat{\psi}^\dagger(\bm{r}),\hat{\psi}^\dagger(\bm{r}')]=0
\ee
$n(r) = \hat{\psi}^\dagger(\bm r)\hat\psi(\bm r)$ is the particle density at the position $\bm r, \hat\psi^\dagger(\bm r)$ is the creation operator of a particle at position $\bm r$, and $\hat\psi(\bm r)$ is the annihilation operator of a particle at position $\bm r$. By promoting the wave functions $\psi(\bm r)$ and $\psi^*(\bm r)$ to operators $\hat\psi(\bm r)$ and $\hat\psi^*(\bm r)$, we regain the picture of wave functions in the three-dimensional physical space, and also the metaphor of the luminous advertisement screen works well. More precisely, the position coordinate $\hat{\bm r}$ is degraded from an operator to a label $\bm r$ defining the position of the light, and instead operators switching the light on $(\hat\psi^\dagger(\bm r))$ and off $(\hat\psi(\bm r))$ emerge. A particle is described as the excitation of a field that is created and annihilated.

We now introduce the phase operator $\hat\theta_n$ describing the interference of a wave by
\be\begin{split}
\hat A_n^\dagger&=(\hat N_n)^{1/2}e^{-\frac{\ii}{\hbar}\hat\theta_n},\\
\hat A_n&=e^{\frac{\ii}{\hbar}\hat\theta_n}(\hat N_n)^{1/2}.
\end{split}\ee
We show that by assuming the canonical conjugation relations of $\hat N$ and $\hat\theta_n$
\be\label{eq1.2.20}
[\hat{N}_n,\hat{\theta}_n]=\ii\hbar
\ee
the commutation relation \eqref{eq1.2.12} is obtained. In order to do so, we define 
\be
\hat N_n(\lambda)=\exp\left(\frac{\ii}{\hbar}\lambda\hat\theta_n\right)\hat N_n\exp\left(-\frac{\ii}{\hbar}\lambda\hat\theta_n\right)
\ee
Then, the following equation holds:
\be\label{eq1.2.22}
[\hat{A}_n,\hat{A}_n^\dagger]=\hat N_n(1)-\hat N_n(0)=\int_0^1\frac{\dd\hat N_n(\lambda)}{\dd\lambda}\dd\lambda. 
\ee
On the other hand, owing to \eqref{eq1.2.20}, we obtain
\be
\frac{\dd\hat N_n(\lambda)}{\dd\lambda}\dd\lambda=\frac{\ii}{\hbar}\exp\left(\frac{\ii}{\hbar}\lambda\hat\theta_n\right)[\hat\theta_n,\hat N_n]\exp\left(-\frac{\ii}{\hbar}\lambda\hat\theta_n\right)=1
\ee
and therefore \eqref{eq1.2.12}.

Now, having introduced the particle number operator $\hat N_n$ and its canonical conjugate, the phase $\hat\theta_n$, it is necessary to stress the following. Obviously, $\hat N_n$ is a Hermitian operator; however, exactly speaking, $\hat\theta_n$ is not Hermitian. In order to see this, we notice that owing to \eqref{eq1.2.22}
\be
\exp\left(\frac{\ii}{\hbar}\lambda\hat\theta_n\right)\hat N_n\exp\left(-\frac{\ii}{\hbar}\lambda\hat\theta_n\right)=\hat N_n+1
\ee
holds, and for a general integer number $m$
\be\begin{split}
\exp\left(\frac{\ii}{\hbar}\lambda\hat\theta_n\right)(\hat N_n)^m\exp\left(-\frac{\ii}{\hbar}\lambda\hat\theta_n\right)&=\left[\exp\left(\frac{\ii}{\hbar}\lambda\hat\theta_n\right)\hat N_n^m\exp\left(-\frac{\ii}{\hbar}\right)\right]^m\\
&=(\hat N_n+1)^m
\end{split}\ee
holds. For a general function $g(N)$ we obtain
\be\label{eq1.2.26}
\exp\left(\frac{\ii}{\hbar}\lambda\hat\theta_n\right)g(\hat N_n)\exp\left(-\frac{\ii}{\hbar}\lambda\hat\theta_n\right)=g(\hat{N}_n+1).
\ee
This means that $\hat U = \exp(\frac{\ii}{\hbar}\hat\theta_n)$ is a linear operator acting on $\hat N_n$, just like $\hat{U}(\bm a)$ in \eqref{eq1.1.60}. If $\hat\theta_n$ were Hermitian, then $\hat U$ would be a unitary operator with $\hat U^\dagger\hat U=\hat U\hat U^\dagger=1$. However, this identity is not true. This can be seen from \eqref{eq1.2.26}: $\hat U^\dagger$ increases $N_n$ by one, $\hat U$ decreases $N_n$ by one. Acting with U on the vacuum state with no particles $|N_n = 0\rangle$ we obtain $\hat U|N_n = 0\rangle = 0$. Acting on this equation with $\hat U^\dagger$ we obtain of course $\hat U^\dagger\hat U|N_n = 0\rangle = 0$. However, because of $\hat U\hat U^\dagger|N_n = 0\rangle = \hat U|N_n = 1\rangle = |N_n =0\rangle \neq 0$ we have just demonstrated that $\hat U^\dagger\hat U\neq\hat U\hat U^\dagger$. Therefore, we conclude that because the particle number $N_n$ is bounded from below, $\hat\theta$ is not Hermitian. However, when only states with $N_n \gg 1$ are considered, the existence of a lower bound can be neglected, and $\hat\theta$ can be regarded to be Hermitian.

Next, we deduce the Hamiltonian occurring after second quantization. Continuing in a heuristic manner as above, we declare in \eqref{eq1.2.8} $\langle\hat H\rangle$ to be an operator again and write
\be\label{eq1.2.27}
\hat H=\sum_{n,m}\hat A_n^\dagger\langle\phi_n|\hat H_1|\phi_m\rangle\hat A_m
\ee

Here, $\hat H_1$ is the single-particle Hamiltonian, being an operator in the sense that it acts on single-particle wave functions $\phi^*(\bm r)$ and $\phi_m(\bm r)$. $\hat H$ is an operator because $\hat A_n^\dagger$ and $\hat A_m$ are operators; however, $\dagger\langle\phi_n|\hat H_1|\phi_m\rangle$ is a simple complex number.

Equation \eqref{eq1.2.27} can also be expressed in terms of the field operators $\hat\psi^\dagger(\bm r)$ and $\hat\psi(\bm r)$:
\be\label{eq1.2.28}
\hat H=\int^3\dd\bm r\hat\psi^\dagger(\bm r)\hat H_1\hat\psi(\bm r). 
\ee

The Heisenberg equation of motion of $\hat\psi$ is given by
\be\label{eq1.2.29}
\ii\hbar\frac{\partial\hat\psi(\bm r, t)}{\partial t}=[\hat\psi(\bm r),\hat H]=\hat H_1\hat\psi(\bm r,t). 
\ee
If $\hat\psi(\bm r)$ were a single-particle wave function, then this equation would be the Schrödinger equation \eqref{eq1.1.1}; however, again we mention that $\hat\psi(\bm r)$ is an operator, and the above equation describes the time evolution of this operator in the Heisenberg picture, which leads to a totally different meaning.

In the framework of second quantization, it is also possible to express the interaction between particles in terms of $\hat\psi(\bm r)$ and $\hat\psi^\dagger(\bm r)$. We mention only the result
\be\label{eq1.2.30}
\sum_{i<j}v(\bm r_i-\bm r_j)\to\frac{1}{2}\int\dd^3\bm r\dd^3\bm r'\hat\psi^\dagger(\bm r)\hat\psi^\dagger(\bm r')v(\bm r-\bm r')\hat\psi(\bm r')\hat\psi(\bm r). 
\ee
The Hamiltonian is then the sum of $\hat H$ in \eqref{eq1.2.28} and the right-hand side of \eqref{eq1.2.30}, and the equation of motion of the field operator is
\be\label{eq1.2.31}
\ii\hbar\frac{\partial\hat\psi(\bm r, t)}{\partial t}=\hat H_1\hat\psi(\bm r,t)+\left[\int\dd^3\bm r'\hat\psi^\dagger(\bm r',t)v(\bm r-\bm r')\hat\psi(\bm r',t)\right]\hat\psi(\bm r,t)
\ee
Comparing this expression with \eqref{eq1.2.29}, we notice that owing to the interaction, a non-linear term emerges. Because this is not the Schrödinger equation, but the Heisenberg equation, there is no conflict with the superposition principle of quantum mechanics.

Finally, we mention the case of fermions. The whole discussion so far is valid for the case when all particles obey Bose statistics. For fermions, all the commutator relations \eqref{eq1.2.12}, \eqref{eq1.2.13}, \eqref{eq1.2.17} and \eqref{eq1.2.18} must be replaced by the anti-commutator relations. By doing so, the Hamiltonian \eqref{eq1.2.28}, \eqref{eq1.2.30} and the equation of motion of the field \eqref{eq1.2.31} are valid as they stand.

In the case when the particles have a spin degree of freedom, the $\bm r$ coordinate must be extended to $(\bm r, \sigma)$ [$\sigma$ is the spin component, for example the eigenvalue $S_z$ of the spin in the $z$ direction). The discussion of the phase of the fermions is not that simple compared with the bosonic case. This question will be examined in Chap. \ref{chap5}.